\documentclass[a4paper]{IEEEtran}

% Ein paar hilfreiche Pakete
\usepackage[utf8]{inputenc}
\usepackage[english]{babel}
\usepackage{graphicx}
\usepackage{amsmath}
\usepackage{amssymb}
\usepackage{mathtools}
\usepackage{subcaption}
\usepackage{hyperref}

\mathtoolsset{showonlyrefs}

\markboth{Proseminar WS 18/19: Anthropomatik: Von der Theorie zur Anwendung}{Proseminar SS 16: Anthropomatik: Von der Theorie zur Anwendung}

% Hier den Titel des eigenen Proseminars eitnragen
\title{Discovery of Processes / Process Minining}

% Hier deinen eigenen Namen
\author{Maximilian Franz}


\begin{document}
\maketitle

% Zusammenfassung
\begin{abstract}
TODO
\end{abstract}

% Erster Abschnitt
\section{Introduction}
Considering the ever increasing amount of data \cite{manyika2011bigdata} \cite{hilbert2011worldcapacity}
the industry is collecting, we need a reliable tool to retrieve information from this data. Keywords like \textit{Big Data} fall more and more often in recent times. The problem in data analysis is not anymore the lack of quality data but the lack of \textit{compute} to process that data into meaningful and actionable information. 

Process Management on the other hand has been a part of business development for quite some time. However, traditional \textit{Process Modelling} has some major drawbacks due to the nature of its approach. 
Errors of traditional \textbf{Process Modelling} include
\begin{itemize}
    \item The model describes an idealized version of reality.
    \item Human behaviour cannot be captured in simple stated forms
    \item The model is at the wrong abstraction level. 
    \item Event Logs only contain positive examples, always.
\end{itemize}
Here, \textit{Process Mining} as a conjunction of \textit{Data Mining} and \textit{Process Modelling} comes in and alleviates some of the problems. 
Process mining can facilitate the construction of better models in less time \cite{process_mining}. Models that are actually based on the empirical reality rather than a simplified model of the world or a biased wish-for. 


\begin{itemize}
    \item The importance and usage of process mining 
    \item The difficulties and problems of process mining (Data-Quality, Limited Perspective)
\end{itemize}

Short outline of the rest of the paper.

% Notation
\section{Process Mining - Task and Terminology - Preliminaries}
\label{sec:terminology}
Introduce important notations in formula and terminology around processes
\begin{itemize}
    \item The Problem Statement after \cite{process_mining}
    \item Process - Case - Event 
    \item Trace - Log - Simple Log - (Ordering Relations) 
    \item Directly-Follow-Graph (maybe only later)
    \item Process Trees and how they can be transformed into any model desired. 
    % \item Shortly introduce Petri-Nets / WF-Nets (for comparisons to other approaches)
\end{itemize}
\textbf{(Maybe put the following in a seperate section.)}


Elaborate on the different quality dimension of Process Mining and quickly summarize different approaches and how they compare in these quality criteria.
\begin{itemize}
    \item Fitness - "replay of logs"
    \item Simplicity - "Occam's razor" 
    \item Generalization - "not overfitting" 
    \item Precision - "not underfitting" 
\end{itemize}
Lead towards 

\section{Inductive Mining}
\label{sec:inductivemining}
Introduce \textit{Inductive Mining} and use the terminology from Section \ref{sec:terminology} to explain \textbf{in detail} how the algorithms work. Starting from a general approach focus on one specific extension of \textit{Inductive Mining} that solves a particular problem from above. (e.g. $IM_F$, $IM_C$, $IM_D$, $IM_{FD}$).

Introduce how \textit{Inductive Mining} allows for the following and why it is "better" compared to other methods described in Section \ref{sec:terminology}
\begin{itemize}
    \item Formal Guarantees
    \item Flexibility 
    \item Scalability
\end{itemize} 
After \cite{process_mining}. How are these properties achieved. 

Introduce and use further formal terminology and important formulas of Inductive Mining along the way. 
\begin{itemize}
    \item Directly Follows Graph
    \item Cuts 
    \item Algorithmic Notation in different notations
\end{itemize}

\subsubsection{Inductive Miner - Infrequent} % (fold)
\label{ssub:inductive_miner_infrequent}

% subsubsection inductive_miner_infrequent (end)

\section{A minimal example}
\label{sec:example}
Showcasing the method with a minimal working example to convey the major advantage. Analysis of the model and its minimal results according to the desired outcomes. 


\section{Discussion \& Related Work}
What are the prospect of \textit{Process Mining}, which techniques are new, which are promising in the context of ever larger amounts of data and new requirements?

\bibliographystyle{ieeetr}
\bibliography{references}
\end{document}
